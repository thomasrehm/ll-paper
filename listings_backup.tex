
\usepackage{listings,xcolor}

\colorlet{punct}{red!60!black}
\definecolor{background}{HTML}{F9F9F9}
\definecolor{delim}{RGB}{20,105,176}
\colorlet{numb}{magenta!60!black}
 \lstset{language=HTML}
\lstdefinelanguage{json}{
    basicstyle=\normalfont\ttfamily,
    numbers=left,
    numberstyle=\scriptsize,
    stepnumber=1,
    numbersep=4pt,
    showstringspaces=false,
    breaklines=true,
    frame=lines,
    backgroundcolor=\color{background},
    literate=
     *{0}{{{\color{numb}0}}}{1}
      {1}{{{\color{numb}1}}}{1}
      {2}{{{\color{numb}2}}}{1}
      {3}{{{\color{numb}3}}}{1}
      {4}{{{\color{numb}4}}}{1}
      {5}{{{\color{numb}5}}}{1}
      {6}{{{\color{numb}6}}}{1}
      {7}{{{\color{numb}7}}}{1}
      {8}{{{\color{numb}8}}}{1}
      {9}{{{\color{numb}9}}}{1}
      {:}{{{\color{punct}{:}}}}{1}
      {,}{{{\color{punct}{,}}}}{1}
      {\{}{{{\color{delim}{\{}}}}{1}
      {\}}{{{\color{delim}{\}}}}}{1}
      {[}{{{\color{delim}{[}}}}{1}
      {]}{{{\color{delim}{]}}}}{1},
}



\subsection{Ablauf Content generierung}

\begin{figure}[htb]
\begin{minipage}[b]{1.0\linewidth}
  \centering
\centerline{\includegraphics[width=\linewidth]{tabelle.png}}
\end{minipage}
\caption{Screenshot Ausschnitt Content Tabelle.}
\label{fig:res}
\end{figure}

\begin{figure}[htb]
\begin{minipage}[b]{1.0\linewidth}
\begin{lstlisting}[language=json,firstnumber=1]
"entry": [{
  {
    [...]
    "gsx$sectionheadline": {
      "$t": "Kreative Aspekte"
    },
    "gsx$articleheadline": {
      "$t": "Tageszeiten und Wetter"
    },
    "gsx$contenttype": {
      "$t": ""
    },
    "gsx$content": {
      "$t": "Wer im Studio arbeitet, ist in der komfortablen Situation, sich nicht nach der Lichtsituation richten zu m\"ussen. Wer aber Natur-, Sport- oder Eventfotografie betreiben will, kommt um Basiswissen \"uber die Sonne und ihre Auswirkungen nicht herum."
    },
    "gsx$internebeschreibung": {
      "$t": ""
    }
  }
}]
\end{lstlisting}

\end{minipage}
\caption{Ausschnitt JSON.}
\label{fig:res}
\end{figure}


\begin{figure}[htb]
\begin{minipage}[b]{1.0\linewidth}
\begin{lstlisting}[language=json]
<section data-linear="1" id="EWTKcQGzuLWnxUi" style="display: block;">
  <h1>Kreative Aspekte</h1>
  <article id="5TrERofVybMWPGC" style="display: block;">
    <div>
      <h2>Tageszeiten und Wetter</h2>
      <p>Wer im Studio arbeitet, ist in der komfortablen Situation, sich nicht nach der Lichtsituation richten zu m\"ussen. Wer aber Natur-, Sport- oder Eventfotografie betreiben will, kommt um Basiswissen \"uber die Sonne und ihre Auswirkungen nicht herum.</p>
    [...]
    </div>
  </article>
</section>
\end{lstlisting}

\end{minipage}
\caption{Ausschnitt fertiges HTML.}
\label{fig:res}
\end{figure}


\begin{figure}[htb]
\begin{minipage}[b]{1.0\linewidth}
  \centering
\centerline{\includegraphics[width=\linewidth]{ergebniss.png}}
\end{minipage}
\caption{Screenshot Ergebnis.}
\label{fig:res}
\end{figure}

